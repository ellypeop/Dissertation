\documentclass{article}

\usepackage{titlesec}
\usepackage{graphicx} % Required for inserting images
\usepackage{biblatex} % Use biblatex for bibliography management
\usepackage{hyperref}

\addbibresource{mybib.bib} % Specify your bibliography file

\titleclass{\subsubsubsection}{straight}[\subsection]

\newcounter{subsubsubsection}[subsubsection]
\renewcommand\thesubsubsubsection{\thesubsubsection.\arabic{subsubsubsection}}
\renewcommand\theparagraph{\thesubsubsubsection.\arabic{paragraph}} % optional; useful if paragraphs are to be numbered

\titleformat{\subsubsubsection}
  {\normalfont\normalsize\bfseries}{\thesubsubsubsection}{1em}{}
\titlespacing*{\subsubsubsection}
{0pt}{3.25ex plus 1ex minus .2ex}{1.5ex plus .2ex}

\makeatletter
\renewcommand\paragraph{\@startsection{paragraph}{5}{\z@}%
  {3.25ex \@plus1ex \@minus.2ex}%
  {-1em}%
  {\normalfont\normalsize\bfseries}}
\renewcommand\subparagraph{\@startsection{subparagraph}{6}{\parindent}%
  {3.25ex \@plus1ex \@minus .2ex}%
  {-1em}%
  {\normalfont\normalsize\bfseries}}
\def\toclevel@subsubsubsection{4}
\def\toclevel@paragraph{5}
%\def\toclevel@paragraph{6}
\def\toclevel@subparagraph{6}
\def\l@subsubsubsection{\@dottedtocline{4}{7em}{4em}}
\def\l@paragraph{\@dottedtocline{5}{10em}{5em}}
\def\l@subparagraph{\@dottedtocline{6}{14em}{6em}}
\makeatother

\setcounter{secnumdepth}{4}
\setcounter{tocdepth}{4}

\title{Dissertation}
\author{Supervisor: Carlos Perez Delgado\\ Programme: MSc Computer Science (Artificial Intelligence) \\ Word Count: 10,000}
\date{December 3, 2024}

\begin{document}

\maketitle

\thispagestyle{empty} % Suppress page number on the title page

% Start page numbering from the next page
\newpage
\setcounter{page}{1}

\section*{Acknowledgments}
% Add the content of the Acknowledgements section here.
\newpage

\section*{Abstract}
% Add the content of the Abstract section here.
\newpage

% List of Figures
\listoffigures
\newpage

% List of Tables
\listoftables
\newpage

% List of Abbreviations (example)
\section*{List of Abbreviations}
\begin{itemize}
    \item AI - Artificial Intelligence
    \item CD - Class Diagram
    \item COBYLA - Constrained Optimization by Linear Approximation
    \item MSc - Master of Science
    \item NISQ - Noisy Intermediate-Scale Quantum
    \item PUB - Primitive Unified Blocs
    \item SD - Sequence Diagram
    \item SDK - Software Development Kit
    \item SU(2) - Special Unitary Group of Degree 2
    \item QUML - Quantum Unified Modelling Language
    \item UML - Unified Modelling Language
    \item VQE - Variational Quantum Eigensolver
    % Add more abbreviations as needed
\end{itemize}
\newpage

\tableofcontents

\newpage

\section{Introduction}

\subsection{Problem Description and Incentive}

Quantum Software Engineering (QSE) is an emerging research field seeking to develop and standardise software engineering principles in quantum technologies. Many of these standards have been adapted from classical software engineering (CSE), attempting to keep them as familiar as possible to CSE whilst being able to distinguish that quantum and classical systems are two fundamentally different hardware. 

One area of interest in QSE is the adaptation of Unified Modelling Language (UML) to model quantum systems and illustrate their communication with classical systems. Two notable papers have introduced adaptations of UML for this purpose: Q-UML \cite{Pérez-Delgado2022} and a quantum UML profile \cite{Pérez-Castillo2022}. Both propose methodologies for incorporating quantum technologies into UML, aiming to broaden the scope of professionals who can contribute to the QSE field and promote early adoption of a standardised quantum modelling language whilst quantum technologies remain in relative infancy. 

The critical question is which quantum UML adaptation offers the best solution. This involves evaluating their effectiveness for modelling quantum systems, their suitability in real-world applications and their potential for widespread adoption. Additionally, should the industry favour full-scale UML modelling or a simplified approach, such as flowcharts, for quantum system design, and which of the two adaptations can best accommodate these design preferences?

\subsection{Goals and Objectives}

This project aims to create and contrast UML diagrams using these two quantum UML approaches.

The first objective is to choose an appropriate real-world example as a starting point for the initial diagrams. The example chosen is the Variational Quantum Eigensolver (VQE), a hybrid quantum/classical algorithm. VQE is ideal for exploring how communication between quantum and classical machines can be represented in a modelling language.

The second objective is to choose the most suitable UML diagram to model the algorithm and compare the quantum UML methods. A sequence diagram was selected because it is one of the most commonly used UML diagrams and illustrates the communication between quantum and classical modules throughout the VQE algorithm.

The third objective is to determine a robust method for comparison and analysis. This will involve combining the author’s observations, following established guidance on good diagram design, and feedback gathered from anonymous polling. Quantum computing and software engineering professionals would be ideal participants in this feedback process.

Finally, the last objective is to create multiple UML diagrams. A valuable exploration will involve modelling fundamental quantum properties such as entanglement and superposition, as proposed by the authors of the quantum UML literature. This approach will enable using other UML diagrams or the application of sequence diagrams in a new context. 

\section{Literature Review}

\subsection{Background}

\subsubsection{Q-SE 2020}

Q-SE 2020 was the first international workshop held virtually in July 2020, co-located with ICSE 2020. Its purpose was to unite researchers and professionals to form a community that could advance quantum software development. 

Paper submissions were requested, including “Towards a Quantum Software Modelling Language” \cite{Perez-Delgado2020}, which laid the groundwork for Q-UML, which was further expanded upon in the paper “A Quantum Software Modelling Language”, presented in “Quantum Software Engineering” \cite{serrano2022quantum}.

The authors of “Quantum Software Engineering” \cite{serrano2022quantum} developed their own quantum UML profile, first published in March 2021 \cite{Pérez-Castillo2021}, which was later expanded on and presented in “Computing” \cite{Pérez-Castillo2022}.

Both papers sought to adapt UML to integrate quantum computing principles. 

\subsubsection{UML 2}

UML 2, simply referred to as UML, is a modelling language and the industry standard for designing and documenting computer systems. It includes 14 types of diagrams divided into structural and behavioural categories. UML 2 signifies that we are currently using the second version of this language. 

As the industry standard for CSE, UML is the logical choice for adaptation for QSE. It should be tested and modelled using current quantum hardware to evaluate its effectiveness. 

Two of the most commonly used diagrams are the sequence diagram in the behavioural category and the class diagram in the structural category.

\subsubsubsection{Sequence Diagram}

Sequence diagrams, part of the behavioural category, belong to a sub-category of interaction diagrams, which includes communication, timing, and interaction overview diagrams.

Sequence diagrams model communication protocols between human and non-human entities. The horizontal axis represents the sequence of communication messages, while the vertical axis indicates the timing of interactions. Each element in a sequence diagram has a lifeline extending vertically, which may terminate if the component is no longer involved in the sequence. 

\subsubsubsection{Class Diagram}

The class diagram is a part of the set of structural diagrams provided in UML. 

\subsubsection{NISQ}

The Noisy Intermediate Scale Quantum (NISQ), known as the near-term quantum computer, employs qubits ranging from tens to hundreds. An NISQ device is sensitive to noise and requires error correction. We are currently in the NISQ era, which means quantum advantage has yet to be achieved.

Despite this, computations can still be performed on NISQ devices, with workloads typically split between classical and quantum computers. One such computation that can be accomplished through this hybrid approach is the Variational Quantum Eigensolver. 

\subsubsection{VQE}

The Variational Quantum Eigensolver (VQE) is a hybrid algorithm that leverages both quantum and classical devices to find the ground state of a given physical system. 

The algorithm starts with the physical system represented as a Hamiltonian and prepares a parametrised trial state called the ansatz. A quantum computer evaluates the ansatz by applying a quantum circuit to explore the parameter search space. A classical optimiser then adjusts the parameters in each iteration, guiding the quantum circuit’s search and gradually refining the ansatz until the algorithm converges on the lowest energy estimate, the ground state of the physical system.

\subsubsection{Qiskit}

Qiskit is an open-source software development kit (SDK) created by IBM to access and utilise their cloud-based quantum computing services. Implemented in Python, Qiskit provides tools and libraries for quantum programming and experimentation. The VQE algorithm represented in the sequence diagram is based on the implementation of VQE in Qiskit \cite{IBM2024}, which interacts directly with the IBM Quantum Platform.

\subsubsubsection{Qiskit and VQE}

IBM Quantum Learning provides online tutorials\cite{Tutorial} on implementing quantum algorithms utilising the Qiskit SDK. The UML diagrams created for this project were based on the tutorial to implement the VQE algorithm in Qiskit\cite{IBM2024}.

The instance \textbf{hamiltonian} is initialised from the class \textbf{SparsePauliOp} by calling the \textit{.from\_list()} method in its construction. The \textbf{hamiltonian} object is a classical representation of Pauli operators, where each operator is given as a string (e.g., "X", "Y", "Z", or "I" for identity). Pauli operators are 2×2 matrices corresponding to spin measurements along the x, y, and z axes\cite{DJORDJEVIC201229}. Each operator string specifies actions on individual qubits. The VQE tutorial uses a 2-qubit system, acting on pairs of Pauli strings called "operator terms". The tensor product of these pairs is assigned a coefficient (represented as a complex number in Python) that defines the strength of each operator term. The linear combination of these terms represents the system's total energy, the Hamiltonian. Only non-zero operators and coefficients are stored, resulting in a sparse representation of operator terms and the Hamiltonian as a whole to reduce computational expense.

The instance \textbf{ansatz} is initialised from the class \textbf{EfficientSU2} with the number of qubits \textbf{hamiltonian} holds passed to it in its construction. The \textbf{EfficientSU2} class provides a hardware-efficient classical representation of a quantum circuit capable of creating parametrised quantum states. The circuit comprises layers of single-qubit operations along with C-NOT gates, which entangle the qubits. The Qiskit documentation defines SU(2) as \textit{"the special unitary group of degree 2, its elements are 2×2 unitary matrices with determinant 1, such as the Pauli rotation gates"}\cite{EfficientSU2}, meaning that the circuit includes layers of operations that rotate the states of individual qubits, specifically using Pauli rotation gates. Each Pauli rotation gate holds a parameter which will be iteratively adjusted to find the lowest energy state of \textbf{ansatz}.

The instance \textbf{backend} is initialised from the class \textbf{QiskitRuntimeService}. The \textbf{QiskitRuntimeService} class interacts with the IBM Qiskit Runtime Service that provides cloud-based access to quantum hardware and quantum simulators. Creating an IBM account and giving a token when executing the code to access the service is necessary. The parameter \textbf{ibm\_quantum} is given to \textbf{backend} during its construction to access the quantum computing platform available on the IBM cloud service. The method \textit{least.busy()} is used to select the next available quantum hardware with the parameter \textbf{simulator} set to false to access the quantum hardware as opposed to a quantum simulator. 

\textbf{QiskitRuntimeService} will instantiate an \textbf{IBMBackend} object which interacts with the selected quantum hardware. The attribute \textbf{target} of the \textbf{IBMBackend} object is accessed and passed as a parameter to the instance \textbf{pm} of the \textbf{StagedPassManager} class, created using the \textit{generate\_preset\_pass\_manager()} method. This process allows the pass manager to receive information regarding the constraints of the selected quantum hardware.

The pass manager will \textit{"define a typical full compilation pipeline from an abstract virtual circuit to one that is optimized and capable of running on the specified backend"}\cite{StagedPassManager}. The \textit{.run()} method is executed on \textbf{pm} to transform \textbf{ansatz} to be compatible with the selected quantum hardware, with the newly transformed ansatz stored in a new variable \textbf{ansatz\_isa}. The \textit{.apply\_layout()} method is then called on \textbf{hamiltonian}, with \textbf{ansatz\_isa} passed as a parameter, to modify its layout to be compatible with the selected quantum hardware. The modified Hamiltonian is then stored as a new variable  \textbf{hamiltonian\_isa}.

A cost function method, defined as \textbf{cost\_func}, is constructed to facilitate access to quantum hardware. It accepts an estimator (written later in the code) and the components of a primitive unified bloc (PUB) as its parameters. The PUB object comprises an array of initial guesses for the parameters of the ansatz in the form of a list (these parameters are also written later in the code), \textbf{hamiltonian\_isa} represented as a list of observables, and a quantum circuit, this being \textbf{ansatz\_isa}. These components form a tuple and are assigned to the variable \textbf{pub}, which is given as a parameter for the estimator object's \textit{.run()} method. The \textit{.result()} method returns a container of PUB results\cite{PrimitiveResult} with slicing utilised to access the estimated energy of \textbf{ansatz\_isa}, which is stored in the variable \textbf{energy}. A dictionary named \textbf{cost\_history\_dict} stores the parameter, iteration, and energy estimate each time the estimator's \textit{.run()} method is executed. Its initial values consist of a placeholder for the parameters, iteration set to zero and an empty list for the energy estimate, with each energy estimate being appended to the list per execution. Finally, the cost function method returns the value of the \textbf{energy} variable as its result after being called.

A random array of initial guess parameters, assigned to the variable \textbf{x0}, is constructed using NumPy’s constant $\pi$ and its \textit{random.random()} method. The code generates an array of random floating-point numbers, scaled to the range of [0,$\pi$2]. The \textbf{ansatz} attribute \textbf{num\_parameters} sets the size of the array, having been stored earlier in the variable \textbf{num\_params}, to match the number of parameters to assign to each of the Pauli rotation gates in the ansatz circuit.

The instance \textbf{session} is initialised from the class \textbf{Session} with \textbf{backend} passed as a parameter to configure it to the selected quantum hardware. The instance \textbf{estimator} is then initialised from the class \textbf{EstimatorV2} with \textbf{session} passed to the estimator's \textbf{mode} attribute. This estimator, which operates within \textbf{cost\_func}, uses \textbf{session} to execute computations on the specified quantum backend. Assigning a  \textbf{Session} object as the estimator's \textbf{mode} facilitates the grouping of iterative calls to the quantum computer\cite{Session} when \textit{estimator.run()} executes, efficiently managing the allocation of jobs to quantum resources.

An instance of the \textbf{EstimatorV2} class is used to interact with Qiskit Runtime Estimator primitive service\cite{EstimatorV2}. Qiskit offers two primary primitives—"Estimator" and "Sampler"—designed to simplify foundational quantum tasks\cite{QiskitRuntime}. The Estimator is essential for VQE as the algorithm’s purpose is to estimate the energy of a system.

The instance \textbf{res} of the classical \textbf{Minimize} function from the SciPy package is initialised with the Constrained Optimization by Linear Approximation (COBYLA) method selected to minimize a scalar function, in this case, the energy estimate returned by \textbf{cost\_func}.  The parameters for \textbf{res} include the \textbf{cost\_func}, the initial guess array \textbf{x0} for the ansatz parameters, along with \textbf{ansatz\_isa}, \textbf{hamiltonian\_isa}, and \textbf{estimator}, all passed from \textbf{res} to \textbf{cost\_func} for execution. 

During the optimisation loop, \textbf{cost\_func} calculates and returns the energy estimate of \textbf{ansatz\_isa}, and \textbf{res} iteratively updates the parameters in \textbf{x0} to reduce this energy estimate. Each call to \textbf{cost\_func} runs 10,000 shots on the quantum circuit via \textit{estimator.run()}. The loop continues until the energy estimate converges to the lowest achievable value, indicating the ground state energy.

Upon completion of the optimisation loop, successful termination of the process is verified by comparing the solution parameter and evaluation count against the stored solution parameter and iteration count within a dictionary maintained in \textbf{cost\_func}. These results are then visualised using the matplotlib package, plotting a graph with the number of iterations on the x-axis and the energy estimates on the y-axis.

\subsection{Related Work}

\subsubsection{Q-UML}

Q-UML is a methodology for integrating quantum computing into the UML standard. The paper establishes a fundamental axiom and five core design principles to consider when designing a UML quantum extension. As the axiom asserts, while the design should adhere to classical standards as closely as possible, it must acknowledge the fundamental differences between classical and quantum hardware. The design principles describe when and how to label components in a diagram as classical or quantum.

In Q-UML, quantum elements are visually distinguished using bold font for textual labels and double lines for diagrammatic components. 

\subsubsection{Quantum UML Profile}

A quantum UML profile is an extension of UML designed explicitly for hybrid information systems incorporating classical and quantum software.

In UML, a profile is a structural diagram that extends the language to support domain-specific custom models. The quantum UML profile introduces a set of stereotypes, tagged values and constraints to represent quantum concepts. 

\section{Design and Implementation}
\subsection{Plant UML}

The initial version of the VQE sequence diagram was drafted using the Plant UML plugin for PyCharm, an open-source tool that enables users to generate UML diagrams from plain text.

To create the Q-UML version of the VQE sequence diagram, a combination of Plant UML and Lucidchart was used to include the double lines and bold text necessary to distinguish quantum components. However, for creating the quantum UML profile, Plant UML alone may suffice, as it should have the tools required to extend UML to represent quantum-specific elements. 

\subsection{Lucidchart}

Lucidchart is a web-based application that creates general-purpose diagrams, including UML diagrams, through a graphical user interface (GUI). It is used when creating Q-UML adaptations as it allows the customisation of elements to contain bold text and double lines necessary to distinguish quantum components. 

The first completed diagram is a PNG file created in Lucidchart. An attempt was made to keep it to a format similar to the Plant UML output. Consistency must be considered to ensure a fair comparison between both diagrams; therefore, even if the VQE quantum UML profile diagram can be reproduced in Plant UML alone, it may need to be translated into Lucidchart to ensure this consistency is upheld.

\subsection{UML Design Choice}

UML offers users flexibility in choosing the level of detail required to represent system information within a diagram. It is essential to strike a balance between including sufficient information for clarity and avoiding over-complication, as UML diagrams should provide a high-level abstraction of the system. This section outlines the general formatting principles applied to all UML diagrams in this project. Subsequent sections on the VQE sequence and class diagrams will delve into specific design choices and their underlying rationale. Much of the reference material for constructing these diagrams has been sourced from the book UML @ Classroom\cite{Seidl_Scholz_Huemer_Kappel_Duffy_2014}. 

All UML diagrams consist of a content area populated by boxes and edges, which together form the specific design of each diagram type. An optional framing element was included in the UML diagrams for this project, enclosing these components within a boundary. The frame header displays the namespace of the system the diagram represents\cite{UMLElementFrame}. For this project, the namespace "Variational Quantum Eigensolver" is used, with abbreviations indicating the type of diagram preceding it. 

A sequence and class diagram have been created to model Qiskit’s implementation of the VQE algorithm. Additional versions of these diagrams have been adapted to illustrate the QUML and Quantum UML Profile extensions.

\subsection{VQE Sequence Diagram}

The elements in the sequence diagram represent instances of classes from Qiskit’s implementation of the VQE algorithm. Each instance is depicted as a rectangle, with a dashed, vertical lifeline extending downward from its creation to connect to a duplicated element at the bottom of the frame, where all the system elements are aligned. The naming convention follows the format \textit{instance:Class}.

Message sequences are simplified where possible to illustrate the provision of an attribute from one instance to another or to illustrate an operation's execution. For example, the message \textit{.num\_qubits} passed from the \textbf{hamiltonian} to the \textbf{ansatz} could have been a sequence of messages where the \textbf{ansatz} first requests the \textbf{.num\_qubits} attribute from the \textbf{hamiltonian} and the \textbf{hamiltonian} provides it as a response message, depicted by a dashed line and open arrowhead. This is more accurate; however, it would result in an over-complicated diagram of many messages. 

All messages in the diagram are synchronous, pointing from a sender to a receiver. A synchronous message indicates that the message must be received before the sender can continue with any further instructions\cite{Seidl_Scholz_Huemer_Kappel_Duffy_2014}. For example, the instance \textbf{pm:PassManager} cannot execute its \textit{pm.run(ansatz)} message until \textbf{backend:QiskitRuntimeService} passes its constraints and optimisation level. A continuous line and a filled triangular arrowhead depict a synchronous message.

A decision was made to omit the creation of an explicit \textbf{IBMBackend} element in the sequence diagram, opting instead to show messages invoking this object as originating from \textbf{backend:QiskitRuntimeService}.
The naming convention effectively conveys that \textbf{QiskitRuntimeService} serves as a "backend" instance. Adding \textbf{IBMBackend} would introduce unnecessary complexity to the sequence diagram, where the emphasis is on message flow rather than object details. These specifics are more appropriately captured in the class diagram, where \textbf{IBMBackend} has been included.

The transformation of \textbf{hamiltonian} and \textbf{ansatz} to \textbf{hamiltonian\_isa} and \textbf{ansatz\_isa} is depicted by a destruction event; a red cross on the lifeline at the point where \textbf{hamiltonian} and \textbf{ansatz} are no longer used in the system.

Activation bars depict the activation of multiple elements within the diagram when an operation is executed\cite{creatley}. This occurs when the pass manager executes its \textit{run.()} method, the  \textbf{hamiltonian} executes \textit{apply\_layout(layout=ansatz\_isa.layout)} and the execution of both \textbf{cost\_func} and \textbf{res}. Multiple elements within the system must be active to execute these operations.

The diagram uses two \textit{loop} fragments and an \textit{alt} fragment. The loop fragment expresses a sequence that is repeatedly executed\cite{Seidl_Scholz_Huemer_Kappel_Duffy_2014} with a boundary encompassing the messages involved in the repeated sequence. In the VQE sequence diagram, an outer loop fragment depicts the repeated exchange of messages between \textbf{cost\_func} and \textbf{res} as they seek to find the lowest energy estimate. An inner loop fragment depicts the repeated runs of the quantum circuit when \textit{estimator.run()} is executed. The upper right corner of the fragment contains a heading of the fragments label and a description of how long the process executes; the outer loop running until \textbf{[lowest energy estimate found]} and the inner loop running \textbf{[10,000 shots]}. The response message from the estimator is shown outside of this loop, as it does not return a result after each shot but once the circuit has completed 10,000 shots. 

The alt fragment in UML represents alternative sequences\cite{Seidl_Scholz_Huemer_Kappel_Duffy_2014}. Here, it evaluates whether the results of the \textbf{cost\_history\_dict} match those from the completed minimisation routine. Based on the boolean outcome—true or false—it determines if the verification is successful or unsuccessful.

\subsubsection{SD QUML}

This diagram has been completed and is attached at the end of this report. It illustrates the sequence of messages between modules necessary to execute the VQE algorithm. 

Multiple Pauli operators define the Hamiltonian “sparsely” instead of creating a complete matrix. It acts upon two qubits at a time, and the qubits count is passed to the ansatz object. The ansatz is a circuit that creates a trial state for the experiment. 

These objects are transformed through a pass manager to be compatible with a quantum computer, the “backend”. This process does not require communication with the quantum hardware, as the information is held classically; therefore, the diagram does not depict these modules as quantum.

A session is created and configured to the same backend, passing multiple circuits to an estimator object while the session handles resource management. This estimator is a quantum object, depicted as such in the diagram, as it utilises quantum hardware to estimate the values of the quantum circuits. 

The methods \texttt{minimize} and \texttt{cost\_func} contain the estimator object and are also depicted as quantum. The \texttt{cost\_func} takes the Hamiltonian, ansatz, and parameters (initially a NumPy array of random values) and combines them into a new variable, \texttt{pub}, which is then given to the estimator object. The estimator executes 10,000 shots, defined during its creation, meaning the quantum circuit runs 10,000 times. This message is marked as a quantum message with \textbf{bold text}, indicating that the communication uses quantum hardware. The estimator then returns an energy estimate, converted to classical information before being received by the \texttt{cost\_func}, therefore depicted as a classical message.

The \texttt{cost\_func} method updates the parameters and records critical information in a \texttt{Dictionary} object per iteration. The \texttt{minimize} method, acting as an outer loop, receives the energy estimate and uses a classical optimiser (COBYLA) to guide the process into finding the lowest energy estimate.

The loop is completed once the energy estimate returned by \texttt{cost\_func} converges to the lowest energy estimate, and the parameters and iterations are verified with the \texttt{Dictionary} object. The final step involves plotting each iteration and energy estimate using the Matplotlib Python library.

\subsubsection{SD Quantum UML Profile}

This has yet to be completed and will be the next step in the process.

\subsection{VQE Class Diagram}

Packages


\subsubsection{CD QUML}
\subsubsection{CD Quantum UML Profile}

\section{Results and Analysis}

\subsection{Author Observations}

My initial observations of the first diagram suggest that while the bold text and double lines used to distinguish quantum components are visible, they are more easily identifiable in examples provided in the Q-UML paper. This is likely due to the larger scope of the VQE sequence diagram and the fact that it is primarily composed of classical elements.
Additional design choices could be considered to better differentiate between classical and quantum components, although these would fall outside UML standards. For example, different colours could represent classical and quantum modules. This feature would only be readily available in applications like Lucidchart but could be incorporated with Plant UML by developing an extension.
Once the quantum UML profile version of the VQE sequence diagram is completed, more detailed observations and comparisons between the two diagrams will be possible.

\subsection{Diagram Literature}

Once both diagrams are completed, it will be crucial to consult relevant literature on effective diagram design to make an informed assessment. Research papers exploring good diagram design, particularly concerning UML, should be considered. An example of such work is the paper "Improving Information System Design: Using UML and Axiomatic Design" \cite{CAVIQUE2022103569}, which provides valuable insights into improving system information design.

\subsection{Polling}

An anonymous questionnaire should be developed to present both VQE sequence diagrams to professionals and researchers in quantum computing and software engineering. Ideally, multiple UML diagrams would be included. The questionnaire should ask participants about the clarity of the diagrams, their understanding of the content being depicted, and their overall observations on the application of UML to quantum technologies. Additionally, gathering information on the industry sectors respondents are associated with would be valuable, providing insight into the diversity of perspectives within the results.

\section{Conclusions}

The paper will conclude with an overall assessment of the advantages and disadvantages of each method. It will address whether the primary goal and all four main objectives were achieved. Additionally, there will be a discussion on whether UML is the most suitable approach for representing quantum systems or if a more general flowchart format might be more appropriate. This will include considering which method could be used in such a case. I assume that Q-UML would remain the most applicable, as the quantum UML profile is inherently based on UML formatting.

\section{Future Work}

Depending on whether the fourth objective is achieved, further exploration into modelling fundamental quantum concepts would be beneficial for evaluating the effectiveness of applying UML to quantum systems.
Plant UML is currently the most convenient tool for creating UML diagrams, as Lucidchart is significantly more labour-intensive. Developing an extension for Plant UML or other open-source tools like Mermaid to incorporate Q-UML formatting would be highly advantageous for future diagram creation. Such an extension would also increase the utility of this method, making it more accessible for continued exploration and broader adoption in the QSE field.

\section*{References}

\section{Bibliography}
\printbibliography

\section{Appendix}

\end{document}
